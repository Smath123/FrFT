% This is part of the TFTB Reference Manual.
% Copyright (C) 1996 CNRS (France) and Rice University (US).
% See the file refguide.tex for copying conditions.


\markright{tfrmh}
\hspace*{-1.6cm}{\Large \bf tfrmh}

\vspace*{-.4cm}
\hspace*{-1.6cm}\rule[0in]{16.5cm}{.02cm}
\vspace*{.2cm}

{\bf \large \fontfamily{cmss}\selectfont Purpose}\\
\hspace*{1.5cm}
\begin{minipage}[t]{13.5cm}
Margenau-Hill time-frequency distribution.
\end{minipage}
\vspace*{.2cm}

{\bf \large \fontfamily{cmss}\selectfont Synopsis}\\
\hspace*{1.5cm}
\begin{minipage}[t]{13.5cm}
\begin{verbatim}
[tfr,t,f] = tfrmh(x)
[tfr,t,f] = tfrmh(x,t)
[tfr,t,f] = tfrmh(x,t,N)
[tfr,t,f] = tfrmh(x,t,N,trace)
\end{verbatim}
\end{minipage}
\vspace*{.3cm}

{\bf \large \fontfamily{cmss}\selectfont Description}\\
\hspace*{1.5cm}
\begin{minipage}[t]{13.5cm}
        {\ty tfrmh} computes the Margenau-Hill distribution of a
        discrete-time signal {\ty x}, or the cross Margenau-Hill
        representation between two signals. This distribution has the
        following expression :
\begin{eqnarray*}
MH_x(t,\nu)&=&\Re\left\{x(t)\ X^*(\nu)\ e^{-j2\pi \nu t}\right\}\\
&=&\int_{-\infty}^{+\infty} \frac{1}{2}\ (x(t+\tau)\ x^*(t)+x(t)\
x^*(t-\tau))\ e^{-j2\pi \nu \tau}\ d\tau.
\end{eqnarray*}
It corresponds to the real part of the Rihaczek distribution (see {\ty
tfrri}).\\

\hspace*{-.5cm}\begin{tabular*}{14cm}{p{1.5cm} p{10cm} c}
Name & Description & Default\\
\hline
        {\ty x}     & signal if auto-MH, or {\ty [x1,x2]} if cross-MH. {\ty
			(Nx=length(x))}\\
        {\ty t}     & time instant(s)          & {\ty (1:Nx)}\\
        {\ty N}     & number of frequency bins & {\ty Nx}\\
        {\ty trace} & if nonzero, the progression of the algorithm is shown
                                         & {\ty 0}\\
     \hline {\ty tfr}   & time-frequency representation\\
        {\ty f}     & vector of normalized frequencies\\

\hline
\end{tabular*}
\vspace*{.1cm}

When called without output arguments, {\ty tfrmh} runs {\ty tfrqview}.
\end{minipage}
\vspace*{.3cm}

{\bf \large \fontfamily{cmss}\selectfont Example}
\begin{verbatim}
         sig=fmlin(128,0.1,0.4); tfrmh(sig,1:128,128,1);
\end{verbatim}
\vspace*{.3cm}

{\bf \large \fontfamily{cmss}\selectfont See Also}\\
\hspace*{1.5cm}
\begin{minipage}[t]{13.5cm}
all the {\ty tfr*} functions.
\end{minipage}
\vspace*{.3cm}

{\bf \large \fontfamily{cmss}\selectfont Reference}\\
\hspace*{1.5cm}
\begin{minipage}[t]{13.5cm}
[1] H. Margenhau, R. Hill ``Correlation between Measurements in Quantum
Theory'', Prog. Theor. Phys. Vol. 26, pp. 722-738, 1961.
\end{minipage}