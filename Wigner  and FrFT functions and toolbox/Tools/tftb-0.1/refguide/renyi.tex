% This is part of the TFTB Reference Manual.
% Copyright (C) 1996 CNRS (France) and Rice University (US).
% See the file refguide.tex for copying conditions.


\markright{renyi}
\hspace*{-1.6cm}{\Large \bf renyi}

\vspace*{-.4cm}
\hspace*{-1.6cm}\rule[0in]{16.5cm}{.02cm}
\vspace*{.2cm}



{\bf \large \fontfamily{cmss}\selectfont Purpose}\\
\hspace*{1.5cm}
\begin{minipage}[t]{13.5cm}
Measure Renyi information.
\end{minipage}
\vspace*{.5cm}


{\bf \large \fontfamily{cmss}\selectfont Synopsis}\\
\hspace*{1.5cm}
\begin{minipage}[t]{13.5cm}
\begin{verbatim}
R = renyi(tfr)
R = renyi(tfr,t)
R = renyi(tfr,t,f)
R = renyi(tfr,t,f,alpha)
\end{verbatim}
\end{minipage}
\vspace*{.5cm}


{\bf \large \fontfamily{cmss}\selectfont Description}\\
\hspace*{1.5cm}
\begin{minipage}[t]{13.5cm}
        {\ty renyi} measures the Renyi information relative to a 2-D
        density function {\ty tfr} (which can be eventually a
        time-frequency representation). Renyi information of order $\alpha$
        is defined as\,:
\begin{eqnarray*}
R_x^{\alpha} = \frac{1}{1-\alpha}\
log_2\left\{\int_{-\infty}^{+\infty}\int_{-\infty}^{+\infty}
\mbox{tfr}_x^{\alpha}(t,\nu)\ dt\ d\nu\right\}
\end{eqnarray*}

The result produced by this measure is expressed in {\it bits} : if one
elementary signal yields zero bit of information ($2^0$), then two well
separated elementary signals will yield one bit of information ($2^1$),
four well separated elementary signals will yield two bits of information
($2^2$), and so on.\\

\hspace*{-.5cm}\begin{tabular*}{14cm}{p{1.5cm} p{8.5cm} c}
Name & Description & Default value\\
\hline
        {\ty tfr} & {\ty (M,N)} 2-D density function (or mass function). Eventually
             {\ty tfr} can be a time-frequency representation, in which case
             its first row must correspond to the lower frequencies\\
        {\ty t} & abscissa vector parametrizing the {\ty tfr} matrix. {\ty t} can be a
            non-uniform sampled vector (eventually a time vector)
                                                & {\ty (1:N)}\\     
        {\ty f} & ordinate vector parametrizing the {\ty tfr} matrix. {\ty f} can be a
            non-uniform sampled vector (eventually a frequency vector)
                                                & {\ty (1:M)}\\      
        {\ty alpha} & rank of the Renyi measure        & {\ty 3}\\
 \hline {\ty R} & the alpha-rank Renyi measure (in bits if {\ty tfr} is a time- 
            frequency matrix).\\
\hline
\end{tabular*}

\end{minipage}

\newpage

{\bf \large \fontfamily{cmss}\selectfont Examples}\\
\hspace*{1.5cm}
\begin{minipage}[t]{13.5cm}

\begin{verbatim}
         s=atoms(64,[32,.25,16,1]); [tfr,t,f]=tfrsp(s); 
         R1=renyi(tfr,t,f,3) 
         ans =
               0.9861

         s=atoms(64,[16,.2,16,1;48,.3,16,1]); [tfr,t,f]=tfrsp(s); 
         R2=renyi(tfr,t,f,3) 
         ans =
               1.9890
         
\end{verbatim}
We can see that if {\ttfamily R} is set to 0 for one elementary atom by
subtracting {\ttfamily R1}, we obtain a result close to 1 bit for two atoms
({\ttfamily R2-R1}=1.0029).
\end{minipage}
\vspace*{.5cm}


{\bf \large \fontfamily{cmss}\selectfont Reference}\\
\hspace*{1.5cm}
\begin{minipage}[t]{13.5cm}
[1] W. Williams, M. Brown, A. Hero III, ``Uncertainty, information and
   time-frequency distributions'', SPIE Advanced Signal Processing
   Algorithms, Architectures and Implementations II, Vol. 1566,
   pp. 144-156, 1991.
\end{minipage}


