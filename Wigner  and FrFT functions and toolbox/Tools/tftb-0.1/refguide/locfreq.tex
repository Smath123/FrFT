% This is part of the TFTB Reference Manual.
% Copyright (C) 1996 CNRS (France) and Rice University (US).
% See the file refguide.tex for copying conditions.



\markright{locfreq}
\hspace*{-1.6cm}{\Large \bf locfreq}

\vspace*{-.4cm}
\hspace*{-1.6cm}\rule[0in]{16.5cm}{.02cm}
\vspace*{.2cm}



{\bf \large \fontfamily{cmss}\selectfont Purpose}\\
\hspace*{1.5cm}
\begin{minipage}[t]{13.5cm}
Frequency localization characteristics.
\end{minipage}
\vspace*{.5cm}


{\bf \large \fontfamily{cmss}\selectfont Synopsis}\\
\hspace*{1.5cm}
\begin{minipage}[t]{13.5cm}
\begin{verbatim}
[fm,B] = locfreq(x)
\end{verbatim}
\end{minipage}
\vspace*{.5cm}


{\bf \large \fontfamily{cmss}\selectfont Description}\\
\hspace*{1.5cm}
\begin{minipage}[t]{13.5cm}
        {\ty locfreq} computes the frequency localization characteristics of
        signal {\ty x}. The definition used for the averaged frequency
        and the frequency spreading are the following\,:
\begin{eqnarray*}
f_m &=& \frac{1}{E_x}\ \int_{-\infty}^{+\infty} \nu\ |X(\nu)|^2\ d\nu\\
B &=& 2\ \sqrt{\frac{\pi}{E_x}\ \int_{-\infty}^{+\infty} (\nu-f_m)^2\
|X(\nu)|^2\ d\nu }
\end{eqnarray*}
where $E_x$ is the energy of the signal and $X(\nu)$ the Fourier transform
of $x(t)$. With this definition (and the one used in {\ty loctime}), the
Heisenberg-Gabor inequality writes $B\ T\geq 1$.\\

\hspace*{-.5cm}\begin{tabular*}{14cm}{p{1.5cm} p{8.5cm} c}
Name & Description & Default value\\
\hline
        {\ty x}     & signal\\
\hline  {\ty fm}    & averaged normalized frequency center\\
        {\ty B}     & frequency spreading\\
\hline
\end{tabular*}

\end{minipage}

\vspace*{1cm}

{\bf \large \fontfamily{cmss}\selectfont Example}
\begin{verbatim}
         z=amgauss(160,80,50).*fmconst(160,0.2);
         [fm,B]=locfreq(z); [fm,B]
         ans = 
               0.2000    0.0200
\end{verbatim}
\vspace*{.5cm}


{\bf \large \fontfamily{cmss}\selectfont See Also}\\
\hspace*{1.5cm}
\begin{minipage}[t]{13.5cm}
\begin{verbatim}
loctime.
\end{verbatim}
\end{minipage}




