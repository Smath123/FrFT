% This is part of the TFTB Reference Manual.
% Copyright (C) 1996 CNRS (France) and Rice University (US).
% See the file refguide.tex for copying conditions.


\markright{momftfr}
\hspace*{-1.6cm}{\Large \bf momftfr}

\vspace*{-.4cm}
\hspace*{-1.6cm}\rule[0in]{16.5cm}{.02cm}
\vspace*{.2cm}

{\bf \large \fontfamily{cmss}\selectfont Purpose}\\
\hspace*{1.5cm}
\begin{minipage}[t]{13.5cm}
Frequency moments (order 1 and 2) of a time-frequency representation.
\end{minipage}
\vspace*{.35cm}

{\bf \large \fontfamily{cmss}\selectfont Synopsis}\\
\hspace*{1.5cm}
\begin{minipage}[t]{13.5cm}
\begin{verbatim}
[tm,T2] = momftfr(tfr)
[tm,T2] = momftfr(tfr,tmin)
[tm,T2] = momftfr(tfr,tmin,tmax)
[tm,T2] = momftfr(tfr,tmin,tmax,time)
\end{verbatim}
\end{minipage}
\vspace*{.35cm}

{\bf \large \fontfamily{cmss}\selectfont Description}\\
\hspace*{1.5cm}
\begin{minipage}[t]{13.5cm}
        {\ty momftfr} computes the frequeny moments of order 1 and 2 of a
        time-frequency representation\,:
\[t_m(f) = \frac{1}{E}\ \int_{-\infty}^{+\infty} t\ \mbox{tfr}(t,f)\ dt\ \
;\ \ T^2(f) = \frac{1}{E}\ \int_{-\infty}^{+\infty} t^2\ \mbox{tfr}(t,f)\
dt - t_m(f)^2.\]

\hspace*{-.5cm}\begin{tabular*}{14cm}{p{1.5cm} p{8.5cm} c} Name &
Description & Default value\\ \hline {\ty tfr} & time-frequency
representation (size {\ty (N,M)}). \\ {\ty tmin} & smallest column
element of {\ty tfr} taken into account & {\ty 1} \\ {\ty tmax} & highest
column element of {\ty tfr} taken into account & {\ty M}\\ {\ty time} &
true time instants & {\ty (1:M)}\\ \hline {\ty tm} & averaged time (first order
moment)\\ {\ty T2} & squared time duration (second order moment)\\
\hline
\end{tabular*}

\end{minipage}
\vspace*{.5cm}

{\bf \large \fontfamily{cmss}\selectfont Example}\\
\hspace*{1.5cm}
\begin{minipage}[t]{13.5cm}
\begin{verbatim}
         sig=fmlin(200,0.1,0.4); [tfr,t,f]=tfrwv(sig); 
         [tm,T2]=momftfr(tfr); 
         subplot(211); plot(f,tm); subplot(212); plot(f,T2); 
\end{verbatim}
The first order moment represents an estimation of the group delay, and the
second order moment the variance of this estimator. We can see that the
estimation is better around the time center position than at the edges of
the observation interval.
\end{minipage}
\vspace*{.35cm}

{\bf \large \fontfamily{cmss}\selectfont See Also}\\
\hspace*{1.5cm}
\begin{minipage}[t]{13.5cm}
\begin{verbatim}
momttfr, margtfr.
\end{verbatim}
\end{minipage}

