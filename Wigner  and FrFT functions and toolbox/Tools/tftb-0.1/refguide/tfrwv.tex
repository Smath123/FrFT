% This is part of the TFTB Reference Manual.
% Copyright (C) 1996 CNRS (France) and Rice University (US).
% See the file refguide.tex for copying conditions.


\markright{tfrwv}
\hspace*{-1.6cm}{\Large \bf tfrwv}

\vspace*{-.4cm}
\hspace*{-1.6cm}\rule[0in]{16.5cm}{.02cm}
\vspace*{.2cm}

{\bf \large \fontfamily{cmss}\selectfont Purpose}\\
\hspace*{1.5cm}
\begin{minipage}[t]{13.5cm}
Wigner-Ville time-frequency distribution.
\end{minipage}
\vspace*{.5cm}

{\bf \large \fontfamily{cmss}\selectfont Synopsis}\\
\hspace*{1.5cm}
\begin{minipage}[t]{13.5cm}
\begin{verbatim}
[tfr,t,f] = tfrwv(x)
[tfr,t,f] = tfrwv(x,t)
[tfr,t,f] = tfrwv(x,t,N)
[tfr,t,f] = tfrwv(x,t,N,trace)
\end{verbatim}
\end{minipage}
\vspace*{.5cm}

{\bf \large \fontfamily{cmss}\selectfont Description}\\
\hspace*{1.5cm}
\begin{minipage}[t]{13.5cm}
        {\ty tfrwv} computes the Wigner-Ville distribution of a
        discrete-time signal {\ty x}, or the cross Wigner-Ville
        representation between two signals. The continuous expression of
        the Wigner-Ville distribution writes
\begin{eqnarray*}
W_x(t,\nu)=\int_{-\infty}^{+\infty} x(t+\tau/2)\ x^*(t-\tau/2)\ e^{-j2\pi
\nu \tau}\ d\tau,   
\end{eqnarray*}
 
\hspace*{-.5cm}\begin{tabular*}{14cm}{p{1.5cm} p{8cm} c}
Name & Description & Default value\\
\hline
        {\ty x}     & signal if auto-WV, or {\ty [x1,x2]} if cross-WV {\ty
			(Nx=length(x))}\\ 
        {\ty t}     & time instant(s)          & {\ty (1:Nx)}\\
        {\ty N}     & number of frequency bins & {\ty Nx}\\
        {\ty trace} & if nonzero, the progression of the algorithm is shown
                                         & {\ty 0}\\
     \hline {\ty tfr}   & time-frequency representation. \\
        {\ty f}     & vector of normalized frequencies\\
 
\hline
\end{tabular*}
\vspace*{.2cm}

When called without output arguments, {\ty tfrwv} runs {\ty tfrqview}.
\end{minipage}
\vspace*{1cm}


{\bf \large \fontfamily{cmss}\selectfont Example}\\
\hspace*{1.5cm}
\begin{minipage}[t]{13.5cm}
The Wigner-Ville distribution is perfectly localized on linear chirp
signals. Here is what we obtain in the discrete case :
\begin{verbatim}
         sig=fmlin(128,0.1,0.4);  
         tfrwv(sig);
\end{verbatim}
\end{minipage}
\vspace*{.5cm}


{\bf \large \fontfamily{cmss}\selectfont See Also}\\
\hspace*{1.5cm}
\begin{minipage}[t]{13.5cm}
all the {\ty tfr*} functions.
\end{minipage}

%\newpage

{\bf \large \fontfamily{cmss}\selectfont References}\\
\hspace*{1.5cm}
\begin{minipage}[t]{13.5cm}
[1] E. Wigner ``On the Quantum Correction for Thermodynamic Equilibrium''
Phys. Res., Vol. 40, pp. 749-759, 1932.\\

[2] J. Ville ``Th�orie et Application de la Notion de Signal Analytique''
C�bles et Transmission, 2eme A. , No. 1, pp. 61-74, 1948.\\

[3] T. Claasen, W. Mecklenbrauker ``The Wigner Distribution - A Tool for
Time-Frequency Signal Analysis'' {\it 3 parts} Philips
J. Res., Vol. 35, No. 3, 4/5, 6, pp. 217-250, 276-300, 372-389, 1980.
\end{minipage}

