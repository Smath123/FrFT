% This is part of the TFTB Reference Manual.
% Copyright (C) 1996 CNRS (France) and Rice University (US).
% See the file refguide.tex for copying conditions.


\markright{ambifuwb}
\hspace*{-1.6cm}{\Large \bf ambifuwb}

\vspace*{-.4cm}
\hspace*{-1.6cm}\rule[0in]{16.5cm}{.02cm}
\vspace*{.2cm}



{\bf \large \fontfamily{cmss}\selectfont Purpose}\\
\hspace*{1.5cm}
\begin{minipage}[t]{13.5cm}
Wide-band ambiguity function.
\end{minipage}
\vspace*{.5cm}


{\bf \large \fontfamily{cmss}\selectfont Synopsis}\\
\hspace*{1.5cm}
\begin{minipage}[t]{13.5cm}
\begin{verbatim}
[waf,tau,theta] = ambifuwb(x)
[waf,tau,theta] = ambifuwb(x,fmin,fmax)
[waf,tau,theta] = ambifuwb(x,fmin,fmax,N)
[waf,tau,theta] = ambifuwb(x,fmin,fmax,N,trace)
\end{verbatim}
\end{minipage}
\vspace*{.5cm}


{\bf \large \fontfamily{cmss}\selectfont Description}\\
\hspace*{1.5cm}
\begin{minipage}[t]{13.5cm}
        {\ty ambifuwb} calculates the asymetric wide-band ambiguity
        function, defined as 
\begin{eqnarray*}
\Xi_x(a,\tau) = \frac{1}{\sqrt{a}}\ \int_{-\infty}^{+\infty} x(t)\
x^*(t/a-\tau)\ dt = \sqrt{a} \int_{-\infty}^{+\infty} X(\nu)\ X^*(a\nu)\
e^{j2\pi a \tau\nu}\ d\nu. 
\end{eqnarray*}

\hspace*{-.5cm}\begin{tabular*}{14cm}{p{1.5cm} p{8.5cm} c}
Name & Description & Default value\\
\hline
        {\ty x}     & signal (in time) to be analyzed (the analytic associated
                signal is considered), of length {\ty Nx} &\\
        {\ty fmin, fmax} & respectively lower and upper frequency bounds of
                the analyzed signal. When specified, these parameters fix
                the equivalent frequency bandwidth (both are expressed in
                Hz)             & {\ty 0, 0.5}\\
        {\ty N}     & number of Mellin points. This number is needed when {\ty fmin}
                and {\ty fmax} are forced     & {\ty Nx}\\
        {\ty trace} & if non-zero, the progression of the algorithm is shown
                                        & 0\\
\hline
        {\ty waf}    & matrix containing the coefficients of the ambiguity
                function. X-coordinate corresponds to the dual variable of 
                scale parameter ; Y-coordinate corresponds to time delay,
                dual variable of frequency.\\
        {\ty tau}   & X-coordinate corresponding to time delay\\
        {\ty theta} & Y-coordinate corresponding to the $\log(a)$ variable,
		where $a$ is the scale\\
\hline
\end{tabular*}
\vspace*{.2cm}

When called without output arguments, {\ty ambifuwb} displays the squared
modulus of the ambiguity function by means of {\ty contour}.
\end{minipage}

\newpage

{\bf \large \fontfamily{cmss}\selectfont Example}\\
\hspace*{1.5cm}
\begin{minipage}[t]{13.5cm}
Consider a BPSK signal (see {\ty anabpsk}) of 256 points, with a keying
period of 8 points, and analyze it with the wide-band ambiguity
function\,:
\begin{verbatim}
         sig=anabpsk(256,8);
         ambifunb(sig);
\end{verbatim}
The result, to be compared with the one obtained with the narrow-band
ambiguity function, presents a thin high peak at the origin of the
ambiguity plane, but with more important sidelobes than with the
narrow-band ambiguity function. It means that the narrow-band assumption is
not very well adapted to this signal, and that the ambiguity in the
estimation of its arrival time and mean frequency is not so small.
\end{minipage}
\vspace*{.5cm}


{\bf \large \fontfamily{cmss}\selectfont See Also}\\
\hspace*{1.5cm}
\begin{minipage}[t]{13.5cm}
\begin{verbatim}
ambifunb.
\end{verbatim}
\end{minipage}

