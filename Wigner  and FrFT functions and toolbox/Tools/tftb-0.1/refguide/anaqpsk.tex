% This is part of the TFTB Reference Manual.
% Copyright (C) 1996 CNRS (France) and Rice University (US).
% See the file refguide.tex for copying conditions.



\markright{anaqpsk}
\hspace*{-1.6cm}{\Large \bf anaqpsk}

\vspace*{-.4cm}
\hspace*{-1.6cm}\rule[0in]{16.5cm}{.02cm}
\vspace*{.2cm}



{\bf \large \fontfamily{cmss}\selectfont Purpose}\\
\hspace*{1.5cm}
\begin{minipage}[t]{13.5cm}
Quaternary Phase Shift Keyed (QPSK) signal.
\end{minipage}
\vspace*{.5cm}


{\bf \large \fontfamily{cmss}\selectfont Synopsis}\\
\hspace*{1.5cm}
\begin{minipage}[t]{13.5cm}
\begin{verbatim}
[y,pm0] = anaqpsk(N)
[y,pm0] = anaqpsk(N,ncomp)
[y,pm0] = anaqpsk(N,ncomp,f0)
\end{verbatim}
\end{minipage}
\vspace*{.5cm}


{\bf \large \fontfamily{cmss}\selectfont Description}\\
\hspace*{1.5cm}
\begin{minipage}[t]{13.5cm}
        {\ty anaqpsk} returns a complex phase modulated signal of
        normalized frequency {\ty f0}, whose phase changes every {\ty
        ncomp} point according to a discrete uniform law, between the
        values {\ty (0, pi/2, pi, 3*pi/2)}.  Such signal is only
        'quasi'-analytic.\\

\hspace*{-.5cm}\begin{tabular*}{14cm}{p{1.5cm} p{8.5cm} c}
Name & Description & Default value\\
\hline
        {\ty N}     & number of points\\
        {\ty ncomp} & number of points of each component & {\ty N/5}\\
        {\ty f0}    & normalized frequency               & {\ty 0.25}\\
  \hline {\ty y}     & signal\\
        {\ty pm0}   & initial phase of each component    \\
\hline
\end{tabular*}

\end{minipage}
\vspace*{1cm}


{\bf \large \fontfamily{cmss}\selectfont Example}
\begin{verbatim}
         [signal,pm0]=anaqpsk(512,64,0.05); 
         subplot(211); plot(real(signal)); 
         subplot(212); plot(pm0);
\end{verbatim}
\vspace*{.5cm}


{\bf \large \fontfamily{cmss}\selectfont See Also}\\
\hspace*{1.5cm}
\begin{minipage}[t]{13.5cm}
\begin{verbatim}
anafsk, anabpsk, anaask.
\end{verbatim}
\end{minipage}
 \vspace*{.5cm}


{\bf \large \fontfamily{cmss}\selectfont Reference}\\
\hspace*{1.5cm}
\begin{minipage}[t]{13.5cm}
[1] W. Gardner {\it Introduction to Random Processes, with Applications to
Signals and Systems}, 2nd Edition, McGraw-Hill, New-York, p. 362 ,1990.  
\end{minipage}

