% This is part of the TFTB Reference Manual.
% Copyright (C) 1996 CNRS (France) and Rice University (US).
% See the file refguide.tex for copying conditions.


\markright{tfrbj}
\hspace*{-1.6cm}{\Large \bf tfrbj}

\vspace*{-.4cm}
\hspace*{-1.6cm}\rule[0in]{16.5cm}{.02cm}
\vspace*{.2cm}

{\bf \large \fontfamily{cmss}\selectfont Purpose}\\
\hspace*{1.5cm}
\begin{minipage}[t]{13.5cm}
Born-Jordan time-frequency distribution.
\end{minipage}
\vspace*{.3cm}

{\bf \large \fontfamily{cmss}\selectfont Synopsis}\\
\hspace*{1.5cm}
\begin{minipage}[t]{13.5cm}
\begin{verbatim}
[tfr,t,f] = tfrbj(x)
[tfr,t,f] = tfrbj(x,t)
[tfr,t,f] = tfrbj(x,t,N)
[tfr,t,f] = tfrbj(x,t,N,g)
[tfr,t,f] = tfrbj(x,t,N,g,h)
[tfr,t,f] = tfrbj(x,t,N,g,h,trace)
\end{verbatim}
\end{minipage}
\vspace*{.5cm}

{\bf \large \fontfamily{cmss}\selectfont Description}\\
\hspace*{1.5cm}
\begin{minipage}[t]{13.5cm}
        {\ty tfrbj} computes the Born-Jordan distribution of a
        discrete-time signal {\ty x}, or the cross Born-Jordan
        representation between two signals. This distribution has the
        following expression :

\[BJ_x(t,\nu)=\int_{-\infty}^{+\infty} \frac{1}{|\tau|}\
\int_{t-|\tau|/2}^{t+|\tau|/2} x(s+\tau/2)\ x^*(s-\tau/2)\ ds\ e^{-j2\pi
\nu \tau} d\tau.\]  

\hspace*{-.5cm}\begin{tabular*}{14cm}{p{1.5cm} p{8cm} c}
Name & Description & Default value\\
\hline
        {\ty x}     & signal if auto-BJ, or {\ty [x1,x2]} if cross-BJ. {\ty
			Nx=length(x)}\\
        {\ty t}     & time instant(s)          & {\ty (1:Nx)}\\
        {\ty N}     & number of frequency bins & {\ty Nx}\\
        {\ty g}     & time smoothing window with odd length, {\ty g(0)}
			being forced to {\ty 1}
                                         & {\ty window(odd(N/10))}\\
        {\ty h}     & frequency smoothing window with odd length, {\ty
			h(0)} being forced to {\ty 1}
                                         & {\ty window(odd(N/4))}\\
        {\ty trace} & if nonzero, the progression of the algorithm is shown
                                         & {\ty 0}\\
     \hline {\ty tfr}   & time-frequency representation\\
        {\ty f}     & vector of normalized frequencies\\
 
\hline
\end{tabular*}
\vspace*{.2cm}

When called without output arguments, {\ty tfrbj} runs {\ty tfrqview}.
\end{minipage}
\vspace*{.5cm}

{\bf \large \fontfamily{cmss}\selectfont Example}
\begin{verbatim}
         sig=fmlin(128,0.05,0.3)+fmlin(128,0.15,0.4);  
         g=window(9,'Kaiser'); h=window(27,'Kaiser'); 
         t=1:128; tfrbj(sig,t,128,g,h,1);
\end{verbatim}
\vspace*{.5cm}

{\bf \large \fontfamily{cmss}\selectfont See Also}\\
\hspace*{1.5cm}
\begin{minipage}[t]{13.5cm}
all the {\ty tfr*} functions.
\end{minipage}
\vspace*{.5cm}


{\bf \large \fontfamily{cmss}\selectfont Reference}\\
\hspace*{1.5cm}
\begin{minipage}[t]{13.5cm}
[1] L. Cohen ``Generalized Phase-Space Distribution Functions'',
J. Math. Phys., Vol. 7, No. 5, pp. 781-786, 1966.
\end{minipage}
