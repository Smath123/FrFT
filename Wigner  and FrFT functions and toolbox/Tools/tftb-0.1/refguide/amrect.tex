% This is part of the TFTB Reference Manual.
% Copyright (C) 1996 CNRS (France) and Rice University (US).
% See the file refguide.tex for copying conditions.



\markright{amrect}
\hspace*{-1.6cm}{\Large \bf amrect}

\vspace*{-.4cm}
\hspace*{-1.6cm}\rule[0in]{16.5cm}{.02cm}
\vspace*{.2cm}



{\bf \large \fontfamily{cmss}\selectfont Purpose}\\
\hspace*{1.5cm}
\begin{minipage}[t]{13.5cm}
Rectangular amplitude modulation.
\end{minipage}
\vspace*{.5cm}


{\bf \large \fontfamily{cmss}\selectfont Synopsis}\\
\hspace*{1.5cm}
\begin{minipage}[t]{13.5cm}
\begin{verbatim}
y = amrect(N)
y = amrect(N,t0)
y = amrect(N,t0,T)
\end{verbatim}
\end{minipage}
\vspace*{.5cm}


{\bf \large \fontfamily{cmss}\selectfont Description}\\
\hspace*{1.5cm}
\begin{minipage}[t]{13.5cm}
        {\ty amrect} generates a rectangular amplitude modulation centered
        on a time {\ty t0}, and with a spread proportional to {\ty T}.
        This modulation is scaled such that {\ty y(t0)=1}.\\

\hspace*{-.5cm}\begin{tabular*}{14cm}{p{1.5cm} p{8.5cm} c}
Name & Description & Default value\\
\hline
        {\ty N}  & number of points\\
        {\ty t0} & time center       &         {\ty N/2}\\
        {\ty T}  & time spreading    &         {\ty 2*sqrt(N)}\\
  \hline {\ty y}  & signal\\
\hline
\end{tabular*}

\end{minipage}
\vspace*{1cm}


{\bf \large \fontfamily{cmss}\selectfont Examples}
\begin{verbatim}
         z=amrect(160);        plot(z);
         z=amrect(160,90,40);  plot(z);
         z=amrect(160,180,70); plot(z);
\end{verbatim}
\vspace*{.5cm}


{\bf \large \fontfamily{cmss}\selectfont See Also}\\
\hspace*{1.5cm}
\begin{minipage}[t]{13.5cm}
\begin{verbatim}
amexpo1s, amexpo2s, amgauss, amtriang.
\end{verbatim}
\end{minipage}



