% This is part of the TFTB Reference Manual.
% Copyright (C) 1996 CNRS (France) and Rice University (US).
% See the file refguide.tex for copying conditions.


\markright{tfrrpwv}
\hspace*{-1.6cm}{\Large \bf tfrrpwv}

\vspace*{-.4cm}
\hspace*{-1.6cm}\rule[0in]{16.5cm}{.02cm}
\vspace*{.2cm}

{\bf \large \fontfamily{cmss}\selectfont Purpose}\\
\hspace*{1.5cm}
\begin{minipage}[t]{13.5cm}
Reassigned  pseudo Wigner-Ville distribution.
\end{minipage}
\vspace*{.5cm}

{\bf \large \fontfamily{cmss}\selectfont Synopsis}\\
\hspace*{1.5cm}
\begin{minipage}[t]{13.5cm}
\begin{verbatim}
[tfr,rtfr,hat] = tfrrpwv(x) 
[tfr,rtfr,hat] = tfrrpwv(x,t) 
[tfr,rtfr,hat] = tfrrpwv(x,t,N) 
[tfr,rtfr,hat] = tfrrpwv(x,t,N,h) 
[tfr,rtfr,hat] = tfrrpwv(x,t,N,h,trace) 
\end{verbatim}
\end{minipage}
\vspace*{.35cm}

{\bf \large \fontfamily{cmss}\selectfont Description}\\
\hspace*{1.5cm}
\begin{minipage}[t]{13.5cm}
        {\ty tfrrpwv} computes the pseudo Wigner-Ville distribution and its
        reassigned version. These distributions are given by the
        following expressions\,:
\begin{eqnarray*}
\hspace*{-1cm}PWV_x(t,\nu;h)&=&\int_{-\infty}^{+\infty} h(\tau)\
x(t+\tau/2)\ x^*(t-\tau/2)\ e^{-j2\pi \nu \tau}\ d\tau\\
\hspace*{-1cm}PWV_x^{(r)}(t',\nu';h)&=&\iint_{-\infty}^{+\infty}
PWV_x(t,\nu;h)\ \delta(t'-\hat{t}(x;t,\nu))\
\delta(\nu'-\hat{\nu}(x;t,\nu))\ dt\ d\nu,
\end{eqnarray*}
where 
\begin{eqnarray*}
\hat{t}(x;t,\nu)= t\ \ \mbox{ and }\ \ 
\hat{\nu}(x;t,\nu)=\nu+j\dfrac{PWV_x(t,\nu;\ens{D}_h)}
{2\pi PWV_x(t,\nu;h)}
\end{eqnarray*}
with $\ens{D}_h(t)=\frac{dh}{dt}(t)$.\\

\hspace*{-.5cm}\begin{tabular*}{14cm}{p{1.5cm} p{8cm} c}
Name & Description & Default value\\
\hline
        {\ty x}     & analyzed signal ({\ty Nx=length(x)}) \\
        {\ty t}     & the time instant(s)      & {\ty (1:Nx)}\\
        {\ty N}     & number of frequency bins & {\ty Nx}\\
        {\ty h}     & frequency smoothing window, {\ty h(0)} being forced to {\ty 1}
                                         & {\ty window(odd(N/4))}\\
        {\ty trace} & if nonzero, the progression of the algorithm is shown
                                         & {\ty 0}\\
     \hline {\ty tfr, rtfr} & time-frequency representation and its reassigned
            version\\
        {\ty hat}   & complex matrix of the reassignment vectors\\

\hline
\end{tabular*}
\vspace*{.2cm}

When called without output arguments, {\ty tfrrpwv} runs {\ty tfrqview}.
\end{minipage}

%\newpage

{\bf \large \fontfamily{cmss}\selectfont Example}
\begin{verbatim}
         sig=fmlin(128,0.1,0.4); 
         h=window(17,'Kaiser'); 
         tfrrpwv(sig,1:128,64,h,1);
\end{verbatim}
\vspace*{.35cm}

{\bf \large \fontfamily{cmss}\selectfont See Also}\\
\hspace*{1.5cm}
\begin{minipage}[t]{13.5cm}
all the {\ty tfr*} functions.
\end{minipage}
\vspace*{.35cm}
{\bf \large \fontfamily{cmss}\selectfont Reference}\\
\hspace*{1.5cm}
\begin{minipage}[t]{13.5cm}
[1] F. Auger, P. Flandrin ``Improving the Readability of Time-Frequency and
Time-Scale Representations by the Reassignment Method'' IEEE Transactions
on Signal Processing, Vol. 43, No. 5, pp. 1068-89, 1995.
\end{minipage}

