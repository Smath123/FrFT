% This is part of the TFTB Reference Manual.
% Copyright (C) 1996 CNRS (France) and Rice University (US).
% See the file refguide.tex for copying conditions.



\markright{plotifl}
\hspace*{-1.6cm}{\Large \bf plotifl}

\vspace*{-.4cm}
\hspace*{-1.6cm}\rule[0in]{16.5cm}{.02cm}
\vspace*{.2cm}



{\bf \large \fontfamily{cmss}\selectfont Purpose}\\
\hspace*{1.5cm}
\begin{minipage}[t]{13.5cm}
Plot normalized instantaneous frequency laws.
\end{minipage}
\vspace*{.5cm}


{\bf \large \fontfamily{cmss}\selectfont Synopsis}\\
\hspace*{1.5cm}
\begin{minipage}[t]{13.5cm}
\begin{verbatim}
plotifl(t,iflaws)
\end{verbatim}
\end{minipage}
\vspace*{.5cm}


{\bf \large \fontfamily{cmss}\selectfont Description}\\
\hspace*{1.5cm}
\begin{minipage}[t]{13.5cm}
        {\ty plotifl} plot the normalized instantaneous frequency 
        laws of each signal component.\\

\hspace*{-.5cm}\begin{tabular*}{14cm}{p{1.5cm} p{8.5cm} c} Name &
Description & Default value\\ \hline {\ty t} & time instants (size {\ty
(M,1)})\\ {\ty iflaws} & {\ty (M,P)}-matrix where each column corresponds
to the instantaneous frequency law of an {\ty (M,1)}-signal. These {\ty P}
signals do not need to be present at the same time instants. The values of
{\ty iflaws} must be between -0.5 and 0.5.\\ \hline
\end{tabular*}

\end{minipage}
\vspace*{1cm}


{\bf \large \fontfamily{cmss}\selectfont Example}
\begin{verbatim}
         N=140; t=0:N-1; [x1,if1]=fmlin(N,0.05,0.3); 
         [x2,if2]=fmsin(70,0.35,0.45,60);
         if2=[zeros(35,1)*NaN;if2;zeros(35,1)*NaN];
         plotifl(t,[if1 if2]);
\end{verbatim}
\vspace*{.5cm}


{\bf \large \fontfamily{cmss}\selectfont See Also}\\
\hspace*{1.5cm}
\begin{minipage}[t]{13.5cm}
\begin{verbatim}
plotsid, tfrqview, tfrview.
\end{verbatim}
\end{minipage}

 
