% This is part of the TFTB Reference Manual.
% Copyright (C) 1996 CNRS (France) and Rice University (US).
% See the file refguide.tex for copying conditions.



\markright{fmpower}
\hspace*{-1.6cm}{\Large \bf fmpower}

\vspace*{-.4cm}
\hspace*{-1.6cm}\rule[0in]{16.5cm}{.02cm}
\vspace*{.2cm}



{\bf \large \fontfamily{cmss}\selectfont Purpose}\\
\hspace*{1.5cm}
\begin{minipage}[t]{13.5cm}
Signal with power-law frequency modulation.
\end{minipage}
\vspace*{.5cm}


{\bf \large \fontfamily{cmss}\selectfont Synopsis}\\
\hspace*{1.5cm}
\begin{minipage}[t]{13.5cm}
\begin{verbatim}
[x,iflaw] = fmpower(N,k,P1)
[x,iflaw] = fmpower(N,k,P1,P2)
\end{verbatim}
\end{minipage}
\vspace*{.5cm}


{\bf \large \fontfamily{cmss}\selectfont Description}\\
\hspace*{1.5cm}
\begin{minipage}[t]{13.5cm}
        {\ty fmpower} generates a signal with a
        power-law frequency modulation :
        \[x(t) = \exp(j2\pi(f_0 t + \frac{c}{1-k} |t|^{1-k})).\] 
 
\hspace*{-.5cm}\begin{tabular*}{14cm}{p{1.5cm} p{8.5cm} c}
Name & Description & Default value\\
\hline
        {\ty N}  & number of points in time\\
        {\ty k}  & degree of the power-law ({\ty k}$\neq$1)\\
        {\ty P1} & if {\ty nargin==3, P1} is a 
            vector containing the two coefficients {\ty (f0 c)} for a
            power-law instantaneous frequency (sampling frequency is set to 1).
            If {\ty nargin=4, P1} (as {\ty P2}) is a time-frequency point of the 
            form {\ty (ti fi)}. {\ty ti} is in seconds and {\ty fi} is a
	    normalized frequency (between 0 and 0.5). The coefficients {\ty f0} 
            and {\ty c} are then deduced such that the frequency modulation 
            law fits the points {\ty P1} and {\ty P2}\\
        {\ty P2} & same as {\ty P1} if {\ty nargin=4}         & optional\\
  \hline {\ty x}  & time row vector containing the modulated signal samples\\
        {\ty iflaw} & instantaneous frequency law\\
 
\hline
\end{tabular*}

\end{minipage}
\vspace*{1cm}


{\bf \large \fontfamily{cmss}\selectfont Example}
\begin{verbatim}
         [x,iflaw]=fmpower(200,0.5,[1 0.5],[180 0.1]);
         subplot(211); plot(real(x));
         subplot(212); plot(iflaw);
\end{verbatim}
\vspace*{.5cm}


{\bf \large \fontfamily{cmss}\selectfont See Also}\\
\hspace*{1.5cm}
\begin{minipage}[t]{13.5cm}
\begin{verbatim}
gdpower, fmconst, fmlin, fmhyp, fmpar, fmodany, fmsin.
\end{verbatim}
\end{minipage}


